\documentclass[12pt,a4paper,twocolumn]{article}

\usepackage[utf8]{inputenc}
\usepackage[T1]{fontenc}

\usepackage{lipsum}
\usepackage{graphicx}
\usepackage{caption} % customizar o label

\makeatletter

% includegraphics maxwidth
\def\maxwidth#1{\ifdim\Gin@nat@width>#1 #1\else\Gin@nat@width\fi}

\begin{document}

\noindent{}\textbf{Exemplo de imagem}\\
\\ usa o package \textbackslash{graphicx}\\

\noindent\\
\lipsum

% exemplo de imagem ocupando as duas colunas
\begin{figure*}[!ht]

	\centering

	\includegraphics[width=\maxwidth{0.8\textwidth},keepaspectratio]{image.png}
	% remove o label automático do latex
	\caption*{\textbf{Imagem 1:} Caption da imagem} 
	

\end{figure*}

\lipsum

% exemplo de imagem inline
\includegraphics[height=1cm,keepaspectratio]{image.png}

\lipsum

% exemplo de imagem onecolumn
\begin{figure}

	\centering

	\includegraphics[width=\maxwidth{0.5\textwidth},keepaspectratio]{image.png}
	% deixa o latex cuidar do label
	\caption{Caption da imagem} 

\end{figure}

\lipsum

\end{document}